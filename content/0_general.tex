\begin{center}\begin{Huge}Analysis II\end{Huge}\end{center}

\section{Allgemein}

\input{content/cos_sin_picture.tex}

\begin{Rechenregeln}{Sinus und Cosinus}{}
    \begin{itemize}
    \item $e^{ix} = \cos(x) + i\cdot \sin(x)$
    \item $\cos(x) = \Re(e^{ix})$
    \item $\sin(x) = \Im(e^{ix})$
    \item $\sin(x\pm y) = \sin(x)\cos(y) \pm \cos(x)\sin(y)$ 
    \item $\cos(x\pm y) = \cos(x)\cos(y) \mp \sin(x)\sin(y)$
    \item $\sin(x+y)\sin(x-y) = \cos^2(y) - \cos^2(x) = \sin^2(x) - \sin^2(y)$
    \item $\cos(x+y)\cos(x-y) = \cos^2(y) - \sin^2(x) = \cos^2(x) - \sin^2(y)$
    \item $\sin{x}\cos{x} = \frac{1}{2}(\sin(x+y) + \sin(x-y))$ 
    \item $\cos{x}\cos{y} = \frac{1}{2}(\cos(x+y) + \cos(x-y))$
    \item  $\sin{x}\sin{y} = \frac{1}{2}(\cos(x-y)-\cos(x+y))$
    \item $\cos(x)^2 + \sin(x)^2 = 1$
    \item $\cos(\pi-x) = -\cos(x)$, $\sin(\pi-x) = \sin(x)$
    \item $\cos(x+\pi) = -\cos(x)$, $\sin(x+\pi) = -\sin(x)$
    \item $\cos(2x) = \cos^2(x) - \sin^2(x) = 1-2\sin^2(x) = 2\cos^2(x) - 1$
    \item $\sin(2x) = 2\sin(x)\cos(x)$
    \item $\tan(2x) = \frac{2\tan(x)}{1-\tan^2(x)}$
    \item $\sin(\frac{x}{2}) = \sqrt{\frac{1-\cos(x)}{2}}$
    \item $\cos(\frac{x}{2}) = \sqrt{\frac{1+\cos(x)}{2}}$
    \item $\tan(\frac{x}{2}) = \frac{1-\cos(x)}{\sin(x)} = \frac{\sin(x)}{1-\cos(x)}$
    \item $\sin^2(x) = \frac{1-\cos(2x)}{2}$
    \item $\cos^2(x) = \frac{1+\cos(2x)}{2}$
    \item $\tan(\pi + x) = \tan(x)$
    \item $\sin(-x) = \sin(x), \cos(-x) = \cos(x), \tan(-x) = -\tan(x)$
    \item Für alle $(a,b)\in \R^2$, sodass $a^2+b^2 = 1$, gibt es $x\in\R$, sodass $a = \cos(x)$, $b = \sin(x)$.
    \item $\sin(x) = \frac{2\tan(x/2)}{1+\tan^2(x/2)}$
    \item $\cos(x) = \frac{1-\tan^2(x/2)}{1+\tan^2(x/2)}$
    \item $\sin(x) = \frac{e^{ix} - e^{-ix}}{2i}$
    \item $\cos(x) = \frac{e^{ix} + e^{-ix}}{2}$
    \item $\int_0^{2\pi} \sin(t)\cdot \cos(t) dt = \int_0^{2\pi} \sin(t) dt = \int_0^{2\pi} \cos(t) dt = 0$
    \item $\int \sin^2(x) dx = \frac{1}{2} (x-\sin(x) \, \cos(x))$
    \item $\int \cos^2(x) dx = \frac{1}{2} (x+\sin(x) \, \cos(x))$
    \item $\int x \sin(x) dx = \sin (x)-x \cos (x)$
    \item $\int x \cos(x) dx = x \sin (x)+\cos (x)$
    \end{itemize}
\end{Rechenregeln}

\begin{Rechenregeln}{Ableitung}{}
    \begin{itemize}
        \item \textbf{Summenregel} $(f(x)+g(x))' = f'(x) + g'(x)$
        \item \textbf{Faktorregel} $(c\cdot f(x))' = c\cdot f'(x)$
        \item \textbf{Produktregel} $(f(x)\cdot g(x))' = f'(x)g(x) + f(x)g'(x)$
        \item \textbf{Quotientenregel} $\left(\frac{f(x)}{g(x)}\right)' = \frac{f'(x)g(x) - f(x)g'(x)}{g^2(x)}(g\neq 0)$
        \item \textbf{Kettenregel} $(f(g(x)))' = (f\circ g)' = f'(g(x))g'(x)$
    \end{itemize}
\end{Rechenregeln}

\begin{Rechenregeln}{Integration}{}
    \begin{itemize}
    \item \textbf{Summe/Differenz:} $\int_a^b (f(x) +/- g(x)) xd = \int_a^b f(x) +/- \int_a^b g(x)$
    \item \textbf{Konstanter Faktor:} $\int_a^b c\cdot f(x)dx = c\cdot \int_a^b f(x)dx$
    \item \textbf{Partielle Integration:} $\int_a^b f'(x)\cdot g(x)dx = \left[f(x)g(x)\right]_a^b - \int_a^b f(x)g'(x)$
    \item \textbf{Substitution:} $\int_{\phi(a)}^{\phi(b)} f(x)dx = \int_a^b f(\phi(t))\phi '(t) dt$
    \item \textbf{$a+c, b+c \in I$} $\int_a^b f(t+c)dt = \int_{a+c}^{b+c} f(x)dx$
    \item \textbf{$ca,cb\in I$: } $\int_a^b f(ct)dt = \frac{1}{c}f(x)dx$
    \item \textbf{Logarithmus: }\;(f stetig diffbar) $\int\frac{f'(t)}{f(t)}dt = \log(|f(x)|)$, bzw. $\int_a^b\frac{f'(t)}{f(t)}dt = \log(f(|b|)) - \log(f(|a|))$
    \end{itemize}
\end{Rechenregeln}

\begin{Rechenregeln}{Limes von Sinus und Cosinus}{}
    \begin{itemize}
    \item $\lim_{x\to 0} \arctan(x) = 0$, $\lim_{x\to\infty} \arctan(x) = \frac{\pi}{2}$
    \item $\lim_{x\to 0} \tan(x) = 0$, $\lim_{x\to\infty} \tan(x) = \infty$, $\lim_{x\to\frac{\pi}{2}} \tan(x) = \infty$
    \item $\lim_{x\to\infty} \cos(x) = [-1, 1]$
    \item $\lim_{x\to\infty} \sin(x) = [-1, 1]$
    \end{itemize}
\end{Rechenregeln}

\begin{Diverses}{}{}
    \begin{itemize}
    \item Kreisgleichung $(x - x_0)^2 + (y - y_0)^2 = r^2$
    \item Ellipsengleichung $\frac{(x-x_0)^2}{a^2} + \frac{(y-y_0)^2}{b^2} = 1$
    \item Mitternachtsformel $x_{1, 2} = \frac{-b \pm \sqrt{b^2 - 4ac}}{2a}$
    \item Matrix Determinante $\begin{vmatrix}
        a & b\\
        c & d
    \end{vmatrix} = ad-bc$
    \item Matrix Invertierbarkeit: Eine quadratische Matrix ist genau dann invertierbar, falls die Determinante $\neq 0$.
    \item Skalarprodukt $x \cdot y = \sum_{i=1}^n x_i y_i$
    \item Kreuzprodukt $a \times b = (a_2b_3-a_3b_2, ~~~ a_3b_1-a_1b_3, ~~~ a_1b_2-a_2b_1)^\top$
    \end{itemize}
\end{Diverses}

\begin{Rechenregeln}{Stammfunktionen, Ableitungen}{}
    $\text{differenzierbar }\implies\text{ stetig }\implies \text{ integrierbar}$\\
    \begin{tabular}{l|l|l}
    $\mathbf{f'(x)}$ & $\mathbf{f(x)}$ & $\mathbf{F(x)}$ \\[0.5em] \hline
    0 & c ($c\in\R)$ & $cx$ \\[0.5em]
    $c$ & $cx$ &$\frac{c}{2}x^2$ \\[0.5em]
    $r\cdot x^{r-1}$ & $x^r (r\in \R \backslash \{-1\}$ & $\frac{x^{r+1}}{r+1}$ \\[0.5em]
    $\frac{-1}{x^2} = -x^{-2}$ & $\frac{1}{x}=x^{-1}$ & $\log |x| $ \\[0.5em]
    $\frac{1}{2 \sqrt{x}} $ & $\sqrt{x}$ & $\frac{2}{3}x^{\frac{3}{2}}$ \\[0.5em]
    $\cos x$ & $\sin x$  & $-\cos x$ \\[0.5em]
    $-\sin x$ & $\cos x$ & $\sin x$\\[0.5em]
    $1+\tan^2x = \frac{1}{\cos^2x}$ & $\tan x$ & $-\log|\cos x|$\\[0.5em]
    $-\frac{1}{\sin^2(x)}$ & $\cot x$ & $\log|\sin x|$\\[0.5em]
    $e^x$ & $e^x$ & $e^x$\\[0.5em]
    $c\cdot e^{cx}$ & $e^{cx}$ & $\frac{1}{c}e^{cx}$\\[0.5em]
    $\log a \cdot a^x$ & $a^x$ & $\frac{a^x}{\log a}$\\[0.5em]
    $\frac{1}{x}$ & $\log|x|$ & $x(\log|x|-1)$ \\[0.5em]
    $\frac{1}{\log a\cdot x}$ & $\log_a |x|$ & $\frac{x}{\log a}(\log|x|-1)$\\[0.5em]
     & & = $x(\log_a|x|-\log_ae)$\\[0.5em]
    $\frac{1}{\sqrt{1-x^2}}$ & $\arcsin x$ & $x\arcsin x + \sqrt{1-x^2}$\\[0.5em]
    $-\frac{1}{\sqrt{1-x^2}}$ & $\arccos x$ &  $x\arccos x - \sqrt{1-x^2}$\\[0.5em]
    $\frac{1}{1+x^2}$ & $\arctan x$ & $x\arctan x - \frac{1}{2}\log(1+x^2)$\\[0.5em]
    $\sinh(x)$ & $\cosh(x)$ & - \\[0.5em]
    $\cosh(x)$ & $\sinh(x)$ & -\\[0.5em]
    $tanh(x)$ & $\frac{1}{cosh^2(x)}$ & -\\[1em]
    $2 \sin(x)\cos(x)$ & $\sin^2(x)$ & $\frac{1}{2}(x-\sin(x)\cos(x))$ \\[.5em]
    $-2\sin(x)\cos(x)$ & $\cos^2(x)$ & $\frac{1}{2}(x+\sin(x)\cos(x))$ \\[.5em]
    $\frac{2 \sin(x)}{\cos^3(x)}$ & $\tan^2(x)$ & $\tan(x) - x$\\[.5em]
    \end{tabular}
\end{Rechenregeln}
